%%%%%%%%%%%%%%%%%%%%%%%%%%%%%%%%%%%%%%%%%
% Beamer Presentation
% LaTeX Template
% Version 1.0 (10/11/12)
%
% This template has been downloaded from:
% http://www.LaTeXTemplates.com
%
% License:
% CC BY-NC-SA 3.0 (http://creativecommons.org/licenses/by-nc-sa/3.0/)
%
%%%%%%%%%%%%%%%%%%%%%%%%%%%%%%%%%%%%%%%%%

%----------------------------------------------------------------------------------------
%	PACKAGES AND THEMES
%----------------------------------------------------------------------------------------

\documentclass[usenames,dvipsnames,table]{beamer}

\mode<presentation> {

\usetheme{Madrid}
%\setbeamertemplate{footline} % To remove the footer line in all slides uncomment this line
%\setbeamertemplate{footline}[page number] % To replace the footer line in all slides with a simple slide count uncomment this line
%\setbeamertemplate{navigation symbols}{} % To remove the navigation symbols from the bottom of all slides uncomment this line
}

\usepackage{graphicx} % Allows including images
\usepackage{booktabs} % Allows the use of \toprule, \midrule and \bottomrule in tables
\usepackage{listings}
\usepackage{enumitem}

%----------------------------------------------------------------------------------------
%	TITLE PAGE
%----------------------------------------------------------------------------------------

\title[ABDA Ex. 2]{Applied Bayesian Data Analysis --- Exercise 3 A}

\author{Kim Albertsson} % Your name
\institute[LTU and CERN]
{
CERN and Luleå University of Technology \\
\medskip
\textit{kim.albertsson@ltu.se}
}
\date{\today}

\newcommand{\cgy}{\cellcolor{gray!25}}
\newcommand{\cgr}{\cellcolor{green!25}}
\newcommand{\cye}{\cellcolor{orange!25}}
\newcommand{\ccb}{\cellcolor{Cerulean!25}}

\begin{document}

\begin{frame}
\titlepage % Print the title page as the first slide
\end{frame}

% \begin{frame}
% \frametitle{Overview} % Table of contents slide, comment this block out to remove it
% \tableofcontents % Throughout your presentation, if you choose to use \section{} and \subsection{} commands, these will automatically be printed on this slide as an overview of your presentation
% \end{frame}

% \begin{frame}
% \frametitle{Multiple Columns}
% \begin{columns}[c] % The "c" option specifies centered vertical alignment while the "t" option is used for top vertical alignment

% \column{.45\textwidth} % Left column and width
% \textbf{Heading}
% \begin{enumerate}
% \item Statement
% \item Explanation
% \item Example
% \end{enumerate}

% \column{.5\textwidth} % Right column and width
% Lorem ipsum dolor sit amet, consectetur adipiscing elit. Integer lectus nisl, ultricies in feugiat rutrum, porttitor sit amet augue. Aliquam ut tortor mauris. Sed volutpat ante purus, quis accumsan dolor.

% \end{columns}
% \end{frame}

%----------------------------------------------------------------------------------------
%	PRESENTATION SLIDES
%----------------------------------------------------------------------------------------
\section{Question 1}
\begin{frame}
\begin{center}
{\huge{Question 1}}
\\\vspace{2em}
What is $p(\mathrm{E}=\mathrm{Blue}|\mathrm{H} \in \{\mathrm{Red}, \mathrm{Blond}\})$?
\end{center}
\end{frame}

\begin{frame}

\textbf{Q:} What is $p(\mathrm{E}=\mathrm{Blue}|\mathrm{H} \in \{\mathrm{Red}, \mathrm{Blond}\})$?

\begin{align}\end{align}

\textbf{Joint probabilities}
\begin{table}
\begin{tabular}{lccccc}
           & Black & Brown & Red  & Blond & $\sum$ \\
     Brown &  0.11 &  0.20 & 0.04 &  0.01 & 0.37  \\
      Blue &  0.03 &  0.14 & 0.03 &  0.16 & 0.36  \\
     Hazel &  0.03 &  0.09 & 0.02 &  0.02 & 0.16  \\
     Green &  0.01 &  0.05 & 0.02 &  0.03 & 0.11  \\
    $\sum$ &  0.18 &  0.48 & 0.12 &  0.21 & 1.0   \\
\end{tabular}
\end{table}
\end{frame}

\begin{frame}

\textbf{Q:} What is $p(\mathrm{E}=\mathrm{Blue}|\mathrm{H} \in \{\mathrm{Red}, \mathrm{Blond}\})$?

\begin{align*}
p(\mathrm{E}=\mathrm{Blue}|\mathrm{H}\in\{\mathrm{Red},\mathrm{Blond}\})
=\\ \frac{\textcolor{green}{p(\mathrm{H}\in\{\mathrm{Red},\mathrm{Blond}\}
                             |\textcolor{green}{\mathrm{E}=\mathrm{Blue}})
                             p(\mathrm{E}=\mathrm{Blue})}}
       {\textcolor{orange}{p(\mathrm{H}\in\{\mathrm{Red}, \mathrm{Blond}\})}}
\end{align*}

\textbf{Joint probabilities}
\begin{table}
\begin{tabular}{lccccc}
           & Black & Brown & Red  & Blond & $\sum$ \\
     Brown &\cgy0.11&\cgy0.20&\cye0.04&\cye0.01& 0.37 \\
      Blue &\cgy0.03&\cgy0.14&\cgr0.03&\cgr0.16& 0.36 \\
     Hazel &\cgy0.03&\cgy0.09&\cye0.02&\cye0.02& 0.16 \\
     Green &\cgy0.01&\cgy0.05&\cye0.02&\cye0.03& 0.11 \\
    $\sum$ &  0.18 &  0.48 & 0.12 &  0.21 & 1.0   \\
\end{tabular}
\end{table}
\end{frame}


\begin{frame}

\textbf{Q:} What is $p(\mathrm{E}=\mathrm{Blue}|\mathrm{H} \in \{\mathrm{Red}, \mathrm{Blond}\})$?

\begin{align*}
p(\mathrm{E}=\mathrm{Blue}|\mathrm{H}\in\{\mathrm{Red},\mathrm{Blond}\})
=\\ \frac{\textcolor{green}{p(\mathrm{H}\in\{\mathrm{Red},\mathrm{Blond}\}
                             |\textcolor{green}{\mathrm{E}=\mathrm{Blue}})
                             p(\mathrm{E}=\mathrm{Blue})}}
       {\textcolor{orange}{p(\mathrm{H}\in\{\mathrm{Red}, \mathrm{Blond}\})}}
=\\ \frac{0.03+0.16}{0.12+0.21}
=\\ 0.58
\end{align*}

\textbf{Joint probabilities}
\begin{table}
\begin{tabular}{lccccc}
           & Black & Brown & Red  & Blond & $\sum$ \\
     Brown &\cgy0.11&\cgy0.20&\cye0.04&\cye0.01& 0.37 \\
      Blue &\cgy0.03&\cgy0.14&\cgr0.03&\cgr0.16& 0.36 \\
     Hazel &\cgy0.03&\cgy0.09&\cye0.02&\cye0.02& 0.16 \\
     Green &\cgy0.01&\cgy0.05&\cye0.02&\cye0.03& 0.11 \\
    $\sum$ &  0.18 &  0.48 & 0.12 &  0.21 & 1.0   \\
\end{tabular}
\end{table}
\end{frame}

\section{Question 2}
\begin{frame}
\begin{center}
{\huge{Question 2}}
\\\vspace{2em}
Create a function to calculate $p(\mathrm{sick}|T)$ where T is any sequence drawn from $\{+, -\}$. Explain how the function operates.
\end{center}
\end{frame}

\begin{frame}
The setting supposes the existence of a disease with prevalence 0.001 in a population ($p(\mathrm{sick})=0.001$). It also supposes the existence of a test for said disease with sensitivity 0.99 ($p(+|\mathrm{sick})=0.99$) and specificity 0.95 ($p(-|\mathrm{sick})=0.05$).

\begin{align}
\end{align}
\end{frame}

\begin{frame}[fragile]
\begin{lstlisting}[language=Python]
p = {}

# Conditionals
p['+|:('] = 0.99 # True positive
p['+|:)'] = 0.05 # False positive
 
 
 
# Marginals
p[':('] = 0.001 # p sick
 
 
 
\end{lstlisting}
\end{frame}

\begin{frame}[fragile]
\begin{lstlisting}[language=Python]
p = {}

# Conditionals
p['+|:('] = 0.99 # True positive
p['+|:)'] = 0.05 # False positive
p['-|:('] = 1 - p['+|:(']
p['-|:)'] = 1 - p['+|:)']

# Marginals
p[':('] = 0.001 # p sick
p[':)'] = 1 - p[':('] # p healthy
p['+'] = p['+|:(']*p[':('] + p['+|:)']*p[':)']
p['-'] = 1 - p['+']
\end{lstlisting}
\end{frame}

\begin{frame}
\begin{align*}
p(h|t_0t_1\ldots) &= p(h|t_0, t_1, \ldots)\\
                  &= \frac{p(t_0, t_1, \ldots|h)}{p(t_0, t_1, \ldots)}p(h) \\
                  &\mathrm{assume\ independence}\\
                  &= \frac{p(t_0|h) p(t_1|h) p(\ldots|h)}{p(t_0)p(t_1)p(\ldots)}p(h)\\
                  &= \ldots \cdot \frac{p(t_1|h)}{p(t_1)}
                            \cdot \frac{p(t_0|h)}{p(t_0)}
                            \cdot p(h)
\end{align*}
\end{frame}

\begin{frame}[fragile]
\begin{lstlisting}[language=Python]
def proba_is_sick(test_scores):
    allowed_symbols = '-+'

    posterior = p[':(']
    for symbol in test_scores:
        assert(symbol in allowed_symbols)
        prior = posterior
        likelihood = p[f'{symbol}|:(']
        evidence = p[f'{symbol}']
        posterior = likelihood / evidence * prior
    return posterior
\end{lstlisting}
\end{frame}

\begin{frame}[fragile]
\begin{lstlisting}[language=Python]
>>> proba_is_sick('')
0.001
>>> proba_is_sick('+')
0.019434628975265017
>>> proba_is_sick('+-')
0.0002047776639544922
\end{lstlisting}
\end{frame}


\end{document} 