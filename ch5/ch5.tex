%%%%%%%%%%%%%%%%%%%%%%%%%%%%%%%%%%%%%%%%%
% Beamer Presentation
% LaTeX Template
% Version 1.0 (10/11/12)
%
% This template has been downloaded from:
% http://www.LaTeXTemplates.com
%
% License:
% CC BY-NC-SA 3.0 (http://creativecommons.org/licenses/by-nc-sa/3.0/)
%
%%%%%%%%%%%%%%%%%%%%%%%%%%%%%%%%%%%%%%%%%

%----------------------------------------------------------------------------------------
%	PACKAGES AND THEMES
%----------------------------------------------------------------------------------------

\documentclass[usenames,dvipsnames,table]{beamer}

\mode<presentation> {

\usetheme{Madrid}
%\setbeamertemplate{footline} % To remove the footer line in all slides uncomment this line
%\setbeamertemplate{footline}[page number] % To replace the footer line in all slides with a simple slide count uncomment this line
\setbeamertemplate{navigation symbols}{} % To remove the navigation symbols from the bottom of all slides uncomment this line
}

\usepackage{amsmath}
\usepackage{graphicx} % Allows including images
\usepackage{booktabs} % Allows the use of \toprule, \midrule and \bottomrule in tables
\usepackage{listings}
\usepackage{xcolor}
\usepackage{xfrac}
% \usepackage{enumitem}

%----------------------------------------------------------------------------------------
%	TITLE PAGE
%----------------------------------------------------------------------------------------

\title[ABDA Ch 5]{Applied Bayesian Data Analysis --- Chapter 5}

\author{Kim Albertsson} % Your name
\institute[LTU and CERN]
{
CERN and Luleå University of Technology \\
\medskip
\textit{kim.albertsson@ltu.se}
}
\date{\today}

\newcommand{\cgy}{\cellcolor{gray!25}}
\newcommand{\cgr}{\cellcolor{green!25}}
\newcommand{\cye}{\cellcolor{orange!25}}
\newcommand{\ccb}{\cellcolor{Cerulean!25}}

\begin{document}

\begin{frame}
\titlepage % Print the title page as the first slide
\end{frame}

% \begin{frame}
% \frametitle{}
% \begin{itemize}
% \item
% \end{itemize}
% \end{frame}

%----------------------------------------------------------------------------------------
%	PRESENTATION SLIDES
%----------------------------------------------------------------------------------------
\section{Chapter 5}
\begin{frame}
\begin{center}
{\huge{Chapter 5}}
\\\vspace{2em}
(Gaining intuition about) Bayes' rule
\vspace{5em}
\end{center}
\end{frame}


\begin{frame}
\frametitle{Introduction}
Bayes' rule is a \emph{central} concept of Bayesian statistics

\begin{align*}
p(c\vert r) &= \frac{p(r\vert c) p(c)}{p(r)} \tag{5.5}
\end{align*}

One primary usecase: Estimate probability of model paramters given data.

\end{frame}



\begin{frame}
\frametitle{Derivation of Bayes' Rule}

\begin{align*}
p(c\vert r) &= \frac{p(r, c)}{p(r)} \tag{5.1} \\
p(r, c) &= p(c\vert r) p(r) \tag{5.2} \\
p(r, c) &= p(r\vert c) p(c) \tag{5.3} \\
p(c\vert r) p(r) &= p(r\vert c) p(c) \tag{5.4}
\end{align*}

Bayes' Rule:
\begin{align*}
p(c\vert r) &= \frac{p(r\vert c) p(c)}{p(r)} \tag{5.5}
\end{align*}

Remember that:
\begin{align*}
p(R=r) &= \int p(R=r, C=c)\, dc \\
p(C=c\vert R=r) &= \frac{p(R=r\vert C=c) p(C=c)}{\int p(R=r, C=c')\, dc'}
\end{align*}

\end{frame}


\begin{frame}
\frametitle{A Note on Notation}
Given a probability distribution $p(r, c)$, what does $p(0.5)$ signify?
Unclear!

\vspace{1em}
More clear alternatives: $p_r(0.5)$ or $p(R=0.5)$

\vspace{1em}
Mathematically there is no problem with: $p(R=c, C=r)$. 
Keep track of your variables and observables!

\end{frame}

\begin{frame}
\frametitle{Bayes' rule intuited from a two-way discrete table}

\begin{align*}
p(c\vert r)  &= \frac{p(r\vert c) p(c)}{p(r)} \tag{5.5} \\
p(\mathrm{E}= \mathrm{Blue}|\mathrm{H}=\mathrm{Red}\})
&= \frac{\textcolor{green}{p(\mathrm{H}=\mathrm{Red}\}
                             |\textcolor{green}{\mathrm{E}=\mathrm{Blue}})
                             p(\mathrm{E}=\mathrm{Blue})}}
       {\textcolor{orange}{p(\mathrm{H}=\mathrm{Red})}}
\end{align*}

\textbf{Joint probabilities}
\begin{table}
\begin{tabular}{lccccc}
           & Black & Brown & Red  & Blond & $\sum$ \\
     Brown &\cgy0.11&\cgy0.20&\cye0.04&\cgy0.01& 0.37 \\
      Blue &\cgy0.03&\cgy0.14&\cgr0.03&\cgy0.16& 0.36 \\
     Hazel &\cgy0.03&\cgy0.09&\cye0.02&\cgy0.02& 0.16 \\
     Green &\cgy0.01&\cgy0.05&\cye0.02&\cgy0.03& 0.11 \\
    $\sum$ &  0.18 &  0.48 & 0.12 &  0.21 & 1.0   \\
\end{tabular}
\end{table}
\end{frame}

\begin{frame}
\frametitle{Example: Test for Disease}

Diagnosis of rare disease.

Suppose one in a thousand has it. $\implies p(\theta=\ddot\frown) = 0.001$
Suppose test with true postitive rate 0.99 $\implies p(T=+\vert \theta=\ddot\frown) = 0.99$
Suppose also false postitive rate 0.05 $\implies p(T=+\vert \theta=\ddot\smile) = 0.05$

\vspace{1em}
Question: Given a postitive test, what is the probability that the subject is ill?

\begin{align*}
p(\theta=\ddot\frown\vert T=+) &= \frac{p(T=+\vert \theta=\ddot\frown) p(\theta=\ddot\frown)}{p(T=+)}
\end{align*}

\begin{align*}
p(\theta=\ddot\smile) &= 1 - p(\theta=\ddot\frown) = 0.999 \\
p(T=+) &= p(T=+\vert \theta=\ddot\frown)p(\theta=\ddot\frown) \\
         & + p(T=+\vert \theta=\ddot\smile)p(\theta=\ddot\smile)
\end{align*}
\end{frame}




\begin{frame}
\frametitle{Applied to Parameters and Data}
Central question: Given some data, how likely are our model parameters?

\begin{align*}
p(\theta\vert X) &= \frac{p(X\vert \theta) p(\theta)}{p(X)}
\end{align*}

Close analogy to the disease example, imagine big table.

\begin{align*}
p(\theta\vert X)      &: \mathrm{posterior} \\
p(X\vert \theta)      &: \mathrm{likelihood} \\
p(\theta)             &: \mathrm{prior} \\
p(X)                  &: \mathrm{evidence} \\
\end{align*}

\end{frame}





\begin{frame}
\frametitle{Data Order Invariance}

$h$ denotes probability of heads.
$t_0t_1\ldots$ is a sequence of coin flips.

\begin{align*}
p(h|t_0t_1\ldots) &= p(h|t_0, t_1, \ldots)\\
                  &= \frac{p(t_0, t_1, \ldots|h)}{p(t_0, t_1, \ldots)}p(h) \\
                  &\mathrm{assume\ independence}\\
                  &= \frac{p(t_0|h) p(t_1|h) p(\ldots|h)}{p(t_0)p(t_1)p(\ldots)}p(h)\\
                  &= \ldots \cdot \frac{p(t_1|h)}{p(t_1)}
                            \cdot \frac{p(t_0|h)}{p(t_0)}
                            \cdot p(h)
\end{align*}

Since multiplication is commutative, the order of the data does not matter.

\end{frame}






\begin{frame}
\frametitle{Complete Example: Estimating Bias in a Coin (I)}

Likelihood for a coin flip:
\begin{align*}
p(t\vert \theta) = \theta^t(1-\theta)^{1-t}; t \in {0, 1}\ \mathrm{Bernoulli\ distribution}
\end{align*}

For several flips:
\begin{align*}
p(T\vert \theta)&= \prod_{t_i \in T} \theta^{t_i}(1-\theta)^{1-t_i} \\
                &= \theta^z(1-\theta)^{N-z}
\end{align*}

where $z$ is total number of heads and $N-z$ is total number of tails.

That is, the probability of heads given a \emph{propensity} for heads.

\end{frame}



\begin{frame}
\frametitle{Complete Example: Estimating Bias in a Coin (II)}
\begin{columns}[c]
\column{.5\textwidth}
Qualitatively: If we assume a coin is most probably fair\ldots

\vspace{1em}
\ldots and measure a single flip resulting in heads, this indicates we should shift our beliefs so heads is more likely.

\column{.5\textwidth}
\begin{figure}
\centering
\includegraphics[width=\linewidth]{img/fig5_1}
\end{figure}
\end{columns}
\end{frame}


\begin{frame}
\frametitle{Influence of sample size on the posterior}
\begin{columns}[c]
\column{.5\textwidth}
A larger sample size makes us less reliant on the prior. Note that the prior is the same in both plots!

\vspace{1em}
This will be derived analytically for specific priors and likelihoods in the next chapter.

\column{.5\textwidth}
\begin{figure}
\centering
\includegraphics[width=\linewidth]{img/fig5_2}
\end{figure}
\end{columns}
\end{frame}

\begin{frame}
\frametitle{Influence of the prior on the posterior}
\begin{columns}[c]
\column{.5\textwidth}
Compare to previous plot.

\vspace{1em}
The prior can be thought of (through data order invariance) as incorporating previous data; A flat prior leans more on the likelihood; A sharp prior does so less.

\vspace{1em}
Assumes the prior is reasonably correct.

\column{.5\textwidth}
\begin{figure}
\centering
\includegraphics[width=\linewidth]{img/fig5_3}
\end{figure}
\end{columns}
\end{frame}




\begin{frame}
\frametitle{Why Bayesian Inference Can Be Difficult}

Bayes rule:
\begin{align*}
p(C=c\vert R=r) &= \frac{p(R=r\vert C=c) p(C=c)}{\int p(R=r, C=c')\, dc'}
\end{align*}

\textbf{Analytic solution:} is not in general easy, or even possible. Exceptions include specific prior-likelihood combinations (likelihood with \emph{conjugate prior}).

Variant: Approximate tricky-to-integrate functions. Called \emph{variational approximation}.

\vspace{1em}
\textbf{Grid approximation:} Discretisize, and numerically solve integrals. Fine for small parameter spaces, quickly runs into the \emph{curse of dimensionality}.

\vspace{1em}
\textbf{Sampling approximation:} Randomly sample the posterior through \emph{Markov-chain Monte-Carlo} methods. These methods skip the evaluation of the integrals. Lead to Bayesian methods gaining practical use.

\end{frame}






\end{document} 