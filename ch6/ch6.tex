%%%%%%%%%%%%%%%%%%%%%%%%%%%%%%%%%%%%%%%%%
% Beamer Presentation
% LaTeX Template
% Version 1.0 (10/11/12)
%
% This template has been downloaded from:
% http://www.LaTeXTemplates.com
%
% License:
% CC BY-NC-SA 3.0 (http://creativecommons.org/licenses/by-nc-sa/3.0/)
%
%%%%%%%%%%%%%%%%%%%%%%%%%%%%%%%%%%%%%%%%%

%----------------------------------------------------------------------------------------
%	PACKAGES AND THEMES
%----------------------------------------------------------------------------------------

\documentclass[usenames,dvipsnames,table]{beamer}

\mode<presentation> {

\usetheme{Madrid}
%\setbeamertemplate{footline} % To remove the footer line in all slides uncomment this line
%\setbeamertemplate{footline}[page number] % To replace the footer line in all slides with a simple slide count uncomment this line
\setbeamertemplate{navigation symbols}{} % To remove the navigation symbols from the bottom of all slides uncomment this line
}

\usepackage{graphicx} % Allows including images
\usepackage{booktabs} % Allows the use of \toprule, \midrule and \bottomrule in tables
\usepackage{fontspec}
\usepackage{listings}
\usepackage{xcolor}
\usepackage{xfrac}
% \usepackage{enumitem}

\usefonttheme[onlymath]{serif}

%----------------------------------------------------------------------------------------
%	TITLE PAGE
%----------------------------------------------------------------------------------------

\title[ABDA Ch 6]{Applied Bayesian Data Analysis --- Chapter 6}

\author{Kim Albertsson} % Your name
\institute[LTU and CERN]
{
CERN and Luleå University of Technology \\
\medskip
\textit{kim.albertsson@ltu.se}
}
\date{\today}

\newcommand{\cgy}{\cellcolor{gray!25}}
\newcommand{\cgr}{\cellcolor{green!25}}
\newcommand{\cye}{\cellcolor{orange!25}}
\newcommand{\ccb}{\cellcolor{Cerulean!25}}

\newcommand{\bern}[2]{{#2}^{#1}({1-#2})^{1-#1}}

\begin{document}

\begin{frame}
\titlepage % Print the title page as the first slide
\end{frame}

% \begin{frame}
% \frametitle{}
% \begin{itemize}
% \item
% \end{itemize}
% \end{frame}

%----------------------------------------------------------------------------------------
%	PRESENTATION SLIDES
%----------------------------------------------------------------------------------------
\section{Chapter 6}
\begin{frame}
\begin{center}
{\huge{Chapter 6}}
\\\vspace{2em}
Inferring a Binomial Probability via Exact Mathematical Analysis
\vspace{5em}
\end{center}
\end{frame}

\begin{frame}
\frametitle{Introduction}

Bayes' rule:
\begin{align*}
p(\theta | \mathbf{y}) = \frac{p(\mathbf{y} | \theta) p(\theta)}
                              {\int p(\mathbf{y} | \theta) p(\theta)\, d\theta}
\end{align*}

Maths made easier if $p(\mathbf{y}|\theta)p(\theta)$ has same form as $p(\theta|\mathbf{y})$. We will consider a reasonable choice for coin flips.

\vspace{1em}
Bernoulli distribution for $p(\mathbf{y}|\theta)$. Beta distribution for $p(\theta)$. The Beta distribution is a \emph{conjugate prior} to the Bernoulli distribution when used as a likelihood.

\end{frame}

\begin{frame}
\frametitle{The Likelihood Function: Bernoulli Distribution}
\begin{columns}[c]
\column{.6\textwidth}
Analytical form for $p(\mathbf{y} | \theta)$: Bernoulli distribution
\begin{align*}
p(\mathbf{y}|\theta) &= \bern{\mathbf{y}}{\theta} \tag{6.1} \\
                     &= p(y_0, y_1, \ldots|\theta) \\
                     &\ \mathrm{assume\ independence}\\
                     &= p(y_0 |\theta) p(y_1|\theta) p(\ldots|\theta) \\
                     &= \prod_{y_i \in \mathbf{y}} \bern{y_i}{\theta} \\
                     &= \theta^{\sum y_i}(1-\theta)^{\sum (1-y_i)} \\
                     &= \theta^{z}(1-\theta)^{N-z} \tag{6.2} \\
\end{align*}

\column{.4\textwidth}
\includegraphics[width=\linewidth]{img/A1c10k}
\end{columns}
\end{frame}





\begin{frame}
\frametitle{A Description of Credibilities: The Beta Distribution}
Analytical form for $p(\theta)$: Beta distribution
\begin{align*}
p(\theta| a, b) &= \operatorname{beta}(\theta| a, b) \\
                &= \frac{\theta^{a-1} (1-\theta)^{b-1}}
                        {B(a, b)}\tag{6.3} \\
                &= \frac{\theta^{a-1} (1-\theta)^{b-1}}
                        {\int_0^1 \theta^{a-1} (1-\theta)^{b-1}\, d\theta} \\
\end{align*}

$B(a, b)$ is the \emph{Beta function}.

\vspace{1em}
Note: Prior and posterior must integrate to $1$. Likelihood needs not.

\end{frame}

\begin{frame}
\frametitle{Beta distribution}
\begin{figure}
\centering
\includegraphics[height=.8\textheight]{img/fig6_1}
\end{figure}
\end{frame}





\begin{frame}
\frametitle{Interlude: Why a Beta Prior? (I)}
Consider the normalised likelihood:
\begin{align*}
\int_0^1 \frac{p(\mathbf{y}|\theta)}{f(\mathbf{y})}\, d\theta &= 1 \\
\int_0^1 \frac{\theta^{z}(1-\theta)^{N-z}}
              {f(\mathbf{y})}\, d\theta &= 1 \tag{using 6.2}
\end{align*}
Consider the beta distribution:
\begin{align*}
\int_0^1 \frac{\theta^{a-1} (1-\theta)^{b-1}}
                        {B(a, b)}\, d\theta &= 1\tag{6.3}
\end{align*}
\end{frame}

\begin{frame}
\frametitle{Interlude: Why a Beta Prior? (II)}

Hence:
\begin{align*}
\int_0^1 \frac{\theta^{z}(1-\theta)^{N-z}}
              {f(\mathbf{y})}\, d\theta
&= \int_0^1 \frac{\theta^{a-1} (1-\theta)^{b-1}}
                         {B(a, b)}\, d\theta
\end{align*}
Clearly $f=B; a=z+1; b=N-z+1$ and one can interpret $\operatorname{beta}(z+1, N-z+1)$ as representing prior knowledge about past coin flips.

\vspace{1em}
\textbf{Note:} $a=1; b=1 \implies z=0; N-z=0$, thus $\operatorname{beta}(1, 1)$ represents \emph{no} prior observations. The book claims a different thing.
\end{frame}



\begin{frame}
\frametitle{Specifying a Beta Prior (I)}

Hence:
\begin{align*}
a=\mu\kappa \ \        &\operatorname{and}\ \ b=(1-\mu)\kappa \tag{6.5} \\
a=\omega(\kappa-2)+1 \ \ &\operatorname{and}\ \ b=(1-\omega)(\kappa-2)+1 \text{\ for\ } \kappa>2 \tag{6.6}
\end{align*}
\begin{align*}
z=\mu\kappa-1 \ \ &\operatorname{and}\ \ N-z=(1-\mu)\kappa-1 \text{\ for\ } \kappa>2 \\
z=\omega(\kappa-2) \ \ &\operatorname{and}\ \ N-z=(1-\omega)(\kappa-2) \text{\ for\ } \kappa>2
\end{align*}
\end{frame}


\begin{frame}
\frametitle{Specifying a Beta Prior (II)}
Can we interpret these numbers in terms of physical quantities?
\begin{align*}
% \mu=\frac{z+1}{\kappa} \ \ &\operatorname{and}\ \ \kappa=\frac{z+1}{\mu} \\
% \omega=\frac{z}{\kappa-2} \ \ &\operatorname{and}\ \ \kappa=\frac{z}{\omega}+2 \text{\ for\ } \kappa>2 \\
% \mu=1-\frac{N-z+1}{\kappa} \ \ &\operatorname{and}\ \ \kappa=\frac{N-z+1}{1-\mu} \\
% \omega=1-\frac{N-z}{\kappa-2} \ \ &\operatorname{and}\ \ \kappa=\frac{N-z}{1-\omega}+2 \text{\ for\ } \kappa>2\\
&\mu=\frac{z+1}{N+2}& \ \text{for } N>0;\ z>0\\
&\omega=\frac{z}{N}& \ \text{for } N>1;\ z>0\\
&\kappa=N+2 & \ \text{for } N>1;\ z>0
\end{align*}

where $\mu$ is the mean, $\omega$ is the mode, and $\kappa$ is the "concentration".

\vspace{1em}
$\mu$ and $\omega$ acually have units of \emph{volume concentration}. $\kappa$ is probably better interpreted as an inverse spread?

\end{frame}

% \begin{frame}
% \frametitle{Specifying a Beta Prior (III)}

% Note that $\kappa$ is the denominator in the calculation for $\mu$.
% \begin{align*}
% \mu = \frac{z+1}{N+2} = \frac{z+1}{\kappa}
% \end{align*}
% Hence $\kappa$ signifies degree of certainty of the dataset.

% Qualitative argument: Suppose $1/\kappa$ represents the standard error squared of the mean $\mu$. Then 
% \begin{align*}
% \frac{1}{\kappa} &= \frac{\sigma_{\mu}^2}{N} \\
% \sigma_{\mu}^2   &= \frac{N}{\kappa} = \frac{N}{N+2}
% \end{align*}
% This would then imply that as the "concentration" $\kappa$ grows, the standard error tends to 1.


% \vspace{1em}
% Caveat: This doesn't make any sense!

% \end{frame}



\begin{frame}
\frametitle{Specifying a Beta Prior (III)}
\begin{figure}
\centering
\includegraphics[height=.8\textheight]{img/fig6_2}
\end{figure}
\end{frame}




\begin{frame}
\frametitle{The Posterior Beta}

Putting the previous work together gives us the posterior:
\begin{align*}
p(\theta| z, N) &= p(z, N) p(\theta) \frac{1}{p(z, N)} \\
                &= \theta^{z}(1-\theta)^{N-z}
                   \frac{\theta^{a-1}(1-\theta)^{b-1}}
                       {B(a,b)} \frac{1}{p(z, N)} \\
                &= \frac{\theta^{z+a-1}(1-\theta)^{N-z+b-1}}
                        {B(a,b)p(z, N)} \\
                &= \frac{\theta^{z+a-1}(1-\theta)^{N-z+b-1}}
                        {B(z+a, N-z+b)}
\end{align*}
\end{frame}




\begin{frame}
\frametitle{Posterior is compromise of prior and likelihood (I)}

Mean of the posterior can be factored:
\begin{align*}
\mu_{\text{posterior}} &= \mu_{\text{likelihood}} w_{\text{likelihood}}
                          \mu_{\text{prior}} w_{\text{prior}}\\
\frac{z+a}{N+a+b} &= \frac{z}{N} \frac{N}{N+a+b} \frac{a}{a+b} \frac{a+b}{N+a+b} \tag{6.9}
\end{align*}

Meaning that the mean of the posterior will be weighted average of the two constituent means. Remember $N_\text{prior}=a+b-2$ (thus this argument is a bit hand-wavy?).
\end{frame}





\begin{frame}
\frametitle{Posterior is compromise of prior and likelihood (II)}
\begin{figure}
\centering
\includegraphics[height=.8\textheight]{img/fig6_3}
\end{figure}
\end{frame}


\begin{frame}
\frametitle{Prior knowledge expressed as a beta distribution}
\begin{figure}
\centering
\includegraphics[width=\textwidth]{img/fig6_4}
\end{figure}
\end{frame}


\begin{frame}
\frametitle{Prior knowledge that cannot be expressed as a beta distribution}
\begin{columns}[c]
\column{.5\textwidth}
Suppose categorical mixture of two betas (e.g. coins from two makers).

\vspace{1em}
Posterior no longer beta (but should converge to one given large enough sample?).
\column{.5\textwidth}
\begin{figure}
\centering
\includegraphics[height=.8\textheight]{img/fig6_5}
\end{figure}
\end{columns}
\end{frame}


\end{document} 